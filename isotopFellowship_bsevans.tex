% Document settings:

\documentclass[12pt]{article}
\linespread{1.6}
\usepackage[margin=1.0in]{geometry}
\usepackage{graphicx}
\usepackage{setspace}

% Header information:

\singlespace \title{\Large \linespread{0} The ``urban-\emph{adapted}'' bird: Does dietary plasticity allow some species of birds to thrive in urban habitats?}
% \author{\normalsize Brian S. Evans} 
\date{}

% Start: 
\begin{document}
\maketitle \vspace{-2cm}

% Background: 

\doublespace
\noindent {\textbf {1. BACKGROUND}}\par
Urbanization during the latter half of the twentieth century has altered habitats, restructured avian communities, and influenced the range sizes and population dynamics of animal species (Mckinney and Lockwood 1999, Marzluff 2001, Chace and Walsh 2006). The unique characteristics of human-built environments confers a conflicting set of benefits and challenges that make it difficult to determine whether urban habitats are advantageous or constitute demographic sinks (Gates and Gysel 1978) for bird populations. As the developed land area in the United States is projected to nearly double between 2000 and 2025 (Alig et al. 2004), understanding how organisms respond to these habitats is of paramount importance. Despite the urgency of the problem for conservation, however, the population level consequences of urbanization on birds remain poorly understood (Shochat et al. 2006). \par

It is hypothesized that anthropogenic habitats may influence bottom-up controls on the distribution and abundance of populations of birds by affecting the resources that determine whether species are able to exist in these habitats (Shochat et al. 2010). Human-dominated environments often contain an abundance of consistent food resources for many species of birds. For example, supplemental feeding by humans, especially for overwintering resident birds, has been found to be positively associated with adult survivorship for a number of granivorous synanthropes (Doherty and Grubb 2002). Moreover, supplemental feeding has been hypothesized to be a primary driver in the northward range expansion of some seed-eating species, such as the Northern Cardinal (Cardinalis cardinalis) because it relaxes winter starvation (Robb et al. 2008). Likewise, human-dominated landscapes are also often dominated by ornamental and invasive species, which have been found to provide essential resources for frugivorous bird species during winter months and thus may facilitate winter survival (e.g., Leston and Rodewald 2006, McCusker et al. 2010). The low temporal variability of resources in anthropogenic habitats is also hypothesized to support higher densities of individuals because they can persist in a state of reduced body condition relative to their rural counterparts (the Credit Card Hypothesis, Shochat 2004). Viewed cumulatively, the abundance and quality of avian food resources is expected to vary spatially along this gradient with resource subsidization as the primary driver of higher food availability at intermediate degrees of urbanization (Raupp et al. 2010).\par

Despite evidence that urban environments can present opposing selection pressures, our understanding of the processes and mechanisms that regulate species persisting in urban landscapes are poorly studied. Research to date on across life history stages (e.g., post-fledging survival, Whittaker and Marzluff 2007, Ausprey and Rodewald 2011 and nestling survival, Ryder et al. 2010, and Adult survival Rodewald and Shustack (2008, 2008a), Evans et al. 2015) have failed to document the negative impact of urbanization. 

% RESEARCH OBJECTIVE: 

\noindent {\textbf {2. RESEARCH OBJECTIVE}}\par
Here, we seek to assess the influence of urban land cover on avian diet and the effects of dietary habits on the health of birds across the rural-to-urban gradient.

The interaction between land use and life history traits of associated species determine whether a given species will be a “winner” (often described urban-adapted species) or “loser” in human-dominated landscapes (described as urban-avoiding species; McKinney and Lockwood 1999; Blair 2001).  Species with specialist niche requirements are expected to be especially sensitive to human-induced habitat modification and may therefore experience high rates of local extinction across the urban habitat matrix (Devictor et al. 2007). For example, obligate insectivores (Lancaster and Rees 1979) may be considerably impacted by even minor modifications of the urban landscape, while species with omnivorous dietary habits are expected to be positive affected by, or even thrive, in urban environments (reviewed in McKinney and Lockwood 1999). 

Availability of winter fruit resources on non-native and ornamental plants may provide relaxation from starvation pressures during winter months when survival of the resident species is dependent on a consistent food supply (Shochat et al. 2010).

% PROPOSED RESEARCH

\noindent {\textbf {3. PROPOSED RESEARCH}}\par
We examine the dietary and stress response of two species of birds: the Northern Cardinal (NOCA, \textit{Cardinalis cardinalis}) and Carolina Chickadee (CACH, \textit{Poecile carolinensis}). Both species exhibit some degree of omnivory, are non-migratory, and utilize bird feeders in suburban and urban habitats. Key differences between the life history of the species include nesting habits (NOCA are cup nesting and CACH cavity nesting species), and response to urbanization (NOCA are found at more urban sites than CACH).

CORT
We would predict that the non-native plants have a negative affect on corticosterone during the summer months, due to a an expected decrease in the number of arthropods and a positive affect during the winter months due to utilization of non-native plant fruits (see Bartusveige and Gorchov 2006).

% GENERAL METHODS

\noindent \underline{GENERAL METHODS}\par

This study will utilize data collected as a part of the Neighborhood Nestwatch Program (NN), a citizen science project run by the Smithsonian Migratory Bird Center. NN has established a network of sampling sites within the Greater Washington D.C. metropolitan area with sampling predominantly located at the homes of project participants. Project participants are accepted into the study based on a wide range of criteria, including level of interest, expected degree of engagement, and position along the rural-to-urban gradient, as assessed by the proportion of impervious surface relative to the impervious surface within our study region and that of existing sites. Habitats represented by study sites range from rural open and forested areas, to suburban and urban environments. Annually, NN technicians visit participant properties during the avian breeding season (between the months of May and August). During each visit, technicians provide an educational banding demonstration for program participants and set two to eight mist nets for a period of three to five hours of mark-recapture. Birds will be captured using target netting with playback of mobbing calls or conspecific song. Technicians measure body mass (to the nearest 0.01 g) on an electronic balance and unflattened wing chord (to the nearest 0.25 mm) with a wing rule. We will age individuals into juvenile and after hatch year (AHY) age classes using plumage, skull ossification, or molt criteria (methodologies vary by species, see Pyle 1997). To limit unwanted variation, only AHY male birds will be sampled. Prior to release, a  tail feather, blood, and claw sample will be taken and individuals will be marked with a US Fish and Wildlife aluminum band and a unique combination of colored plastic bands.\par

Corticosterone and isotope concentrations in feather, blood, and claw samples represent the stress and diet response of birds, respectively, at different points in a birds' annual cycle. Tail feathers are molted immediately following the end of the breeding cycle and thus CORT, $\delta^{13}$C, and $\delta^{15}$N levels represent a snapshot of the diet and health of the bird during the growth of the feather, a period of roughly two weeks (citation). Claw samples, for which only the tip of the claw will be removed, are grown continuously and are reprentative of bird diet and health during the non-breeding period (winter). Blood samples provide a measure of corticosterone concentrations and isotopic ratios related to the stress and diet immediately prior to the time of capture, during breeding season. By combining each of these points in a birds' annual cycle, we are able to 



\par

% HYPOTHESES AND PREDICTIONS

\noindent \underline{HYPOTHESES AND PREDICTIONS}\par

% HYPOTHESIS 1 plasticity

\noindent{\textbf{Hypothesis 1. The dietary habits of conspecific birds vary in composition and breadth along the rural-to-urban gradient.}} Across bioregions, urban environments have been shown to favor dietary generalists and granivores and exclude birds with insectivorous diets. Indeed, in an assessment of avian community composition across the rural-to-urban gradient in metropolitan Washington, D.C. we observed a sharp decline in the proportional abundance of obligate insectivores (Evans et al. 2015). (Note: Bearhop 2004 suggested that isotopes can be a proxy for niche width)\par

% PREDICTION 1.1 niche position

\noindent \underline{Prediction 1.1}: \textit{The ratios of $\delta^{13}$C and $\delta^{15}$N in feathers will reflect dietary shifts from insect to plant-based and anthropogenic food sources.}  Analysis of d13C and d15N in feathers (and possible diet items including bird food?) will give us a sense of the the proportional composition of plant-based foods (I’m hoping we can pick up a distinct signature for bird seed) and insects in the diet. We expect that with increasing urbanization diets will become more omnivorous and the ratio of insects to plant-based foods (including anthropogenic food sources) will decline. To test whether increased omnivory is advantageous to bird health, we use corticosterone as a proxy of avian condition. This last bit is the awkward one — however, survival was higher for NOCA in urban habitats and we assumed that this was driven by resources. Using enhanced survival of NOCA as our guide, we might expect that a species exhibiting high plasticity (NOCA) would benefit from this dietary shift while those exhibiting low plasticity (e.g., CACH) would by negatively influenced by a shift from insectivory.\par

% PREDICTION 1.2 niche width

\noindent \underline{Prediction 1.2}: \textit{ The dietary niche width among conspecifics, as measured as variance in $\delta^{13}$C and $\delta^{15}$N in feathers,  will increase with increasing urban land cover.} Populations with higher degrees of dietary specialization are expected to exhibit low variance in $\delta$-values (Bearhop et al. 2004). CChange this bit, see Newsome's paper -- Shannon-Weiner metric ... Further, we predict that the dietary niche width among NOCA will be greater than that of CACH. \par

% PREDICTION 1.3 CORT with niche width and position

\noindent \underline{Prediction 1.3}: \textit{Avian condition, as measured by corticosterone concentrations in feathers, decreases with increasing omnivory for NOCA but not CACH.} We expect ... To examine the adult condition of birds, we will measure the stress hormone corticosterone (CORT) in feathers using a methanol based extraction, modified from Bortolotti et al. (2008), and a commercial ELISA kit (Corticosterone ELISA kit; Neogen Corporation, Ayr, UK) following Carbajal et al. (2014). The concentration of CORT in feathers represents stress exposure during the time in which the feather was grown, a period of up to three weeks. Feather CORT has been found to be predictive of several biologically important indicators of bird health including breeding success (Crossin et al. 2013), body condition (Harms et al. 2010), and survival (Koren et al. 2012). We will analyze CORT concentrations as a function of neighborhood and yard-scale habitat and management practices within a generalized linear model framework, using model selection to determine the variables most predictive of avian condition (see Burnham and Anderson 2002). Preliminary analysis of CORT in Carolina Chickadee (Poecile carolinensis) feathers at NN sites (n = 22) exhibit elevated CORT concentrations in sites dominated by non-native plants ($\beta$ = -3.2 $\pm$ 1.4, p-value = 0.04) but no response to the degree of urbanization at a given site (proportion of impervious surface within 500 m, $\beta$ = 0.2 $\pm$ 0.3, p-value = 0.47). These results suggest that CORT concentrations may be highly reflective of the biological response of birds to yard-scale habitat features.\par

% HYPOTHESIS 2 credit debit

\noindent{\textbf{Hypothesis 2. Urban habitats minimize starvation pressure by providing a more consistent temporal distribution of resources.}} This is a test of Shochat’s credit-debt hypothesis that we mentioned in the survivorship paper. Shochat suggested that the more natural environments have greater temporal variation which would increase starvation pressures. Under this scenario anthropogenic food sources and winter fruiting of non-native species would enhance survival. Hobson (1999) suggested that d15N enrichment is representative of nutritional stress. To test this, we would have to sample birds during the non breeding season (perhaps early spring?) and expect higher d15N ratios in rural habitats (relative to more urban ones). To test whether there is an influence of nutritional stress on bird health, we use corticosterone as a proxy of condition. \par

% PREDICTION 2.1 starvation pressure (enhanced d15N)

\noindent \underline{Prediction 2.1}: \textit{Birds will exhibit enhanced levels of $\delta^{15}$N, as measured in avian blood, along the rural-to-urban gradient.} 
text lorem ipsum yada yada. \par

% PREDICTION 2.2 CORT with starvation pressure

\noindent \underline{Prediction 2.2}: \textit{Concentrations of corticosterone in avian blood will increase proportionally with $\delta^{15}$N.} \par \par

We will use Structural Equation Modeling (SEM; McCune and Grace 2002) to examine hypothesized relationships between environmental attributes, place meaning, place attachment, and place-enhancing behaviors (Fig. 3). SEM is a multivariate 
technique that allows us to simultaneously analyze the complex, and often correlated, inter-relationships among measured (e.g., socioeconomics and land-cover) and latent variables (i.e., those estimated from measured values such as 
environmental attributes and place attachment) that comprise a coupled human-natural system. We will use a Bayesian framework to develop and test a priori SEM in JAGS, implemented in Program R (see Plummer et al. 2003). We will evaluate 
the influence of structural linkages using Markov chain Monte Carlo sampling and use Bayesian model selection to compare a priori models (see Raferty 1993). SEMs will include socio-economic factors since these characteristics might 
influence maintenance decisions, as well as personal histories (e.g., prior/other nature experiences from open-ended interviews) and the amount of urban land and tree cover in the neighborhood surrounding participants’ households (as 
described in the Research Overview section above). With SEM, we will distinguish the relative influence of environmental and sociocultural attributes on management decisions and the linkages among environmental/sociocultural attributes, 
place meanings, place attachments, and place-enhancing behaviors (Fig. 3). The SEM for Objective 3 will be incorporated into the SEM for the overall project model (see Synthesis section below).

% PROJECT RELEVANCE

\noindent {\textbf {4. RELEVANCE OF THE PROPOSED RESEARCH}}: 
My previous research has addressed the influence of urbanization on avian dispersal (dissertation citation), community composition (dissertation citation), and survival (Ryder et al. 2010, Evans et al. 2015) in metropolitan Washington, D.C. Despite emergent patterns that have clearly demonstrated the influence of urban habitat on these biological processes, the mechanisms driving these patterns remain largelly unknown. By addressing the distribution and influence of resources in avian diet, the proposed research project builds on my previous work by addressing the mechanism by which urbanization is expected to have the greatest impact on bird populations.

% ROLE OF SMITHSONIAN

\noindent {\textbf {5. ROLE OF THE SMITHSONIAN INSTITUTION}}:
Research in urban environments is often limited by accessibility and adequate coverage of the urban gradient (Cooper et al. 2007). To alleviate this problem we will utilized sites (n = 242) that were part of Neighborhood Nestwatch (hereafter referred to as “NN”), which is an ongoing citizen science project run by the Smithsonian Migratory Bird Center. NN is the ideal sampling framework for studying the effects of urbanization on avian demography because it provides access to residential properties within core urban and suburban environments as well as forested and agricultural land cover types (Figure 3.1). By incorporating privately owned land within our study design, we are able to capture portions of the urban and suburban matrix not normally monitored in avian survivorship studies.

% WORK CITED 
\noindent \begin{center} {\textbf {WORKS CITED}}\end{center} \par


\end{document}

