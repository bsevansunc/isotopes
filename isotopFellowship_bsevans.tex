% Document settings:

\documentclass[12pt]{article}
\linespread{1.6}
\usepackage[margin=1.0in]{geometry}
\usepackage{graphicx}
\usepackage{setspace}

% Header information:

\singlespace \title{\Large \linespread{0} The ``urban-\emph{adapted}'' bird: Does dietary plasticity allow some species of birds to thrive in urban habitats?}
% \author{\normalsize Brian S. Evans} 
\date{}

% Start: 
\begin{document}
\maketitle \vspace{-2cm}

% Background: 

\doublespace
\noindent {\textbf {1. BACKGROUND}}\par
Urbanization during the latter half of the twentieth century has altered habitats, restructured avian communities, and influenced the range sizes and population dynamics of animal species (Mckinney and Lockwood 1999, Marzluff 2001, 
Chace and Walsh 2006). The unique characteristics of human-built environments confers a conflicting set of benefits and challenges that make it difficult to determine whether urban habitats are advantageous or constitute demographic 
sinks (Gates and Gysel 1978) for bird populations. As the developed land area in the United States is projected to nearly double between 2000 and 2025 (Alig et al. 2004), understanding how organisms respond to these habitats is of 
paramount importance. Despite the urgency of the problem for conservation, however, the population level consequences of urbanization on birds remain poorly understood (Shochat et al. 2006). \par

% RESEARCH OBJECTIVE: 

\noindent {\textbf {2. RESEARCH OBJECTIVE}}\par
Here, we seek to assess the influence of urban land cover on avian diet and the effects of dietary habits on the health of birds across the rural-to-urban gradient.

% PROPOSED RESEARCH

\noindent {\textbf {3. PROPOSED RESEARCH}}\par
We examine the dietary and stress response of two species of birds: the Northern Cardinal (NOCA, \textit{Cardinalis cardinalis}) and Carolina Chickadee (CACH, \textit{Poecile carolinensis}). Both species exhibit some degree of omnivory, 
are non-migratory, and utilize bird feeders in suburban and urban habitats. Key differences between the life history of the species include differences in foraging strata (NOCA forage on the ground while CACH are foliage gleaners), 
nesting habits (NOCA are cup nesting and CACH cavity nesting species), and response to urbanization (NOCA are found at more urban sites than CACH).

% HYPOTHESIS 1 plasticity

\noindent{\textbf{Hypothesis 1. The dietary habits of conspecific birds vary in composition and breadth along the rural-to-urban gradient. }} \par
Across bioregions, urban environments have been shown to favor dietary generalists and granivores and exclude birds with insectivorous diets. Indeed, in an assessment of avian community composition across the rural-to-urban gradient in 
metropolitan Washington, D.C. we observed a sharp decline in the proportional abundance of obligate insectivores (Evans et al. 2015). (Note: Bearhop 2004 suggested that isotopes can be a proxy for niche width)\par

% PREDICTION 1.1 niche position

\noindent \underline{Prediction 1.1}: \textit{The ratios of $\delta^{13}$C and $\delta^{15}$N in feathers will reflect dietary shifts from insect to plant-based and anthropogenic food sources.}  Analysis of d13C and d15N in feathers (and possible diet items including bird food?) will give us a sense of the the proportional composition of plant-based foods (I’m hoping we can pick up a distinct signature for bird seed) and insects in the diet. We expect that with increasing urbanization diets will become more omnivorous and the ratio of insects to plant-based foods (including anthropogenic food sources) will decline. To test whether increased omnivory is advantageous to bird health, we use corticosterone as a proxy of avian condition. This last bit is the awkward one — however, survival was higher for NOCA in urban habitats and we assumed that this was driven by resources. Using enhanced survival of NOCA as our guide, we might expect that a species exhibiting high plasticity (NOCA) would benefit from this dietary shift while those exhibiting low plasticity (e.g., CACH) would by negatively influenced by a shift from insectivory.\par

% PREDICTION 1.2 niche width

\noindent \underline{Prediction 1.2}: \textit{ The dietary niche width among conspecifics, as measured as variance in the ratios of $\delta^{13}$C and $\delta^{15}$N in feathers,  will increase with increasing urban land cover.} Change this bit, 
see Newsome's paper -- Shannon-Weiner metric \\

% PREDICTION 1.3 CORT with niche width and position

\noindent \underline{Prediction 1.3}: \textit{Avian condition, as measured by corticosterone concentrations in feathers, decreases with increasing omnivory for NOCA but not CACH.} We expect ... To examine the adult condition of birds, we will 
measure the stress hormone corticosterone (CORT) in feathers using a methanol based extraction, modified from Bortolotti et al. (2008), and a commercial ELISA kit (Corticosterone ELISA kit; Neogen Corporation, Ayr, UK) following 
Carbajal et al. (2014). The concentration of CORT in feathers represents stress exposure during the time in which the feather was grown, a period of up to three weeks. Feather CORT has been found to be predictive of several biologically 
important indicators of bird health including breeding success (Crossin et al. 2013), body condition (Harms et al. 2010), and survival (Koren et al. 2012). We will analyze CORT concentrations as a function of neighborhood and yard-scale 
habitat and management practices within a generalized linear model framework, using model selection to determine the variables most predictive of avian condition (see Burnham and Anderson 2002). Preliminary analysis of CORT in Carolina 
Chickadee (Poecile carolinensis) feathers at NN sites (n = 22) exhibit elevated CORT concentrations in sites dominated by non-native plants ($\beta$ = -3.2 $\pm$ 1.4, p-value = 0.04) but no response to the degree of urbanization at a 
given site (proportion of impervious surface within 500 m, $\beta$ = 0.2 $\pm$ 0.3, p-value = 0.47). These results suggest that CORT concentrations may be highly reflective of the biological response of birds to yard-scale habitat 
features.\par

% HYPOTHESIS 2 credit debit

\noindent{\textbf{Hypothesis 2. Urban habitats minimize starvation pressure by providing a more consistent temporal distribution of resources.}} \\ 
This is a test of Shochat’s credit-debt hypothesis that we mentioned in the survivorship paper. Shochat suggested that the more natural environments have greater temporal variation which would increase starvation pressures. Under this 
scenario anthropogenic food sources and winter fruiting of non-native species would enhance survival. Hobson (1999) suggested that d15N enrichment is representative of nutritional stress. To test this, we would have to sample birds 
during the non breeding season (perhaps early spring?) and expect higher d15N ratios in rural habitats (relative to more urban ones). To test whether there is an influence of nutritional stress on bird health, we use corticosterone as a 
proxy of condition. \par

% PREDICTION 2.1 starvation pressure (enhanced d15N)

\noindent \underline{Prediction 2.1}: \textit{Birds will exhibit enhanced levels of $\delta^{15}$N, as measured in avian blood, along the rural-to-urban gradient.} 
text lorem ipsum yada yada. \par

% PREDICTION 2.2 CORT with starvation pressure

\noindent \underline{Prediction 2.2}: \textit{Concentrations of corticosterone in avian blood will increase proportionally with $\delta^{15}$N.} \par \par

We will use Structural Equation Modeling (SEM; McCune and Grace 2002) to examine hypothesized relationships between environmental attributes, place meaning, place attachment, and place-enhancing behaviors (Fig. 3). SEM is a multivariate 
technique that allows us to simultaneously analyze the complex, and often correlated, inter-relationships among measured (e.g., socioeconomics and land-cover) and latent variables (i.e., those estimated from measured values such as 
environmental attributes and place attachment) that comprise a coupled human-natural system. We will use a Bayesian framework to develop and test a priori SEM in JAGS, implemented in Program R (see Plummer et al. 2003). We will evaluate 
the influence of structural linkages using Markov chain Monte Carlo sampling and use Bayesian model selection to compare a priori models (see Raferty 1993). SEMs will include socio-economic factors since these characteristics might 
influence maintenance decisions, as well as personal histories (e.g., prior/other nature experiences from open-ended interviews) and the amount of urban land and tree cover in the neighborhood surrounding participants’ households (as 
described in the Research Overview section above). With SEM, we will distinguish the relative influence of environmental and sociocultural attributes on management decisions and the linkages among environmental/sociocultural attributes, 
place meanings, place attachments, and place-enhancing behaviors (Fig. 3). The SEM for Objective 3 will be incorporated into the SEM for the overall project model (see Synthesis section below).

% PROJECT RELEVANCE

\noindent {\textbf {4. RELEVANCE OF THE PROPOSED RESEARCH}}\par
Lorem ipsum yada yada yada

% ROLE OF SMITHSONIAN

\noindent {\textbf {5. ROLE OF THE SMITHSONIAN INSTITUTION}}\par
Lorem ipsum yada yada yada

% WORK CITED 
\noindent \begin{center} {\textbf {WORKS CITED}}\end{center} \par


\end{document}

