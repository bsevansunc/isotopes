% Document settings:

\documentclass[12pt]{article}
\linespread{1.6}
\usepackage[margin=1.0in]{geometry}
\usepackage{graphicx}
\usepackage{setspace}
\usepackage{wasysym}

% Header information:

\singlespace \title{\Large \linespread{0} Work Cited}
% \author{\normalsize Brian S. Evans} 
\date{}

% Start: 
\begin{document}
\maketitle \vspace{-2cm}

% WORK CITED 
\singlespace
\hangindent=2em

\noindent 1. McKinney, M.L. and J.L. Lockwood. 1999. Biotic homogenization: a few winners replacing many losers in the next mass extinction. Trends in Ecology & Evolution 14: 450-453.}\\ \\
2. Marzluff, J.M., R. Bowman and R.E. Donnelly. 2001. Avian Ecology and Conservation in an Urbanizing World. Kluwer Academic Publishers, Norwell, MA.\\ \\
3. Chace, J.F. and J.J. Walsh. 2006. Urban effects on native avifauna: a review. Landscape and urban planning 74: 46-69.\\ \\
4. U.S. Census Bureau. 2012. Statistical abstracts of the United States. GPO, Washington, D.C., USA.  \\ \\
5. Czech, B., and P. R. Krausman. 1997. Distribution and causation of species endangerment in the United States. Science 277:1116�1117. \\ \\
6. Lawler, J.J., D. J. Lewis, E. Nelson, A.J. Plantinga, S. Polasky, J.C. Withey, D.P. Helmers, S. Martinuzzi, D. Pennington, and V.C. Radeloff. 2014. Projected land-use change impacts on ecosystem services in the United States PNAS 111: 7492-7497. \\ \\
7. Shochat, E. P.S. Warren, S.H. Faeth, N.E. McIntryre and D. Hope. 2006. From patterns to emerging processes in mechanistic urban ecology. Trends in Ecology and Evolution 21: 2-7. \\ \\
8. Blair RB. 2001. Birds and butterflies along urban gradients in two ecoregions of the U.S. Pages. 3356. in Lockwood JL, McKinney ML, eds. Biotic Homogenization. Norwell (MA): Kluwer. \\ \\
9. Shochat, E., S. Lerman, and E. Fern�ndez-Juricic. 2010. Birds in urban ecosystems: Population dynamics, community structure, biodiversity, and conservation. Urban Ecosystem Ecology: 75�86.  \\ \\
10. Evans, B.S., T.B. Ryder, Reitsma, R., Hurlbert, A.H., and P.P. Marra. 2014. Characterizing avian survival along a rural-to-urban land use gradient. Ecology 96: 1631�1640. \\ \\
11. Lancaster R.K. and W.E. Rees. 1979. Bird communities and the structure of urban habitats. Canadian Journal of Zoology. 57: 2358-2368. \\ \\
12. Devictor, V., R. Julliard, D. Couvet, A. Lee, and F. Jiguet. 2007. Functional homogenization effect of urbanization on bird communities. Conservation Biology 21: 741-751. \\ \\
13. Robb, G. N., R.A. McDonald, D.E. Chamberlain, and S. Bearhop. 2008. Food for thought: supplementary feeding as a driver of ecological change in avian populations. Frontiers in Ecology and the Environment 6: 476-484.\\ \\
14. McCusker, C.E., M.P. Ward, and J.D. Brawn. 2010. Seasonal responses of avian communities to invasive bush honeysuckles (Lonicera spp.). Biological Invasions 12: 2459-2470.  \\ \\
15. Tallamy, D. W. 2004. Do alien plants reduce insect biomass? Conservation Biology 18: 1689-1692. \\ \\
16. Blair, R.B. and E.M. Johnson. 2008. Suburban habitats and their role for birds in the urban�rural habitat network: points of local invasion and extinction? Landscape Ecology 23: 1157-1169. \\ \\
17. Shochat, E., 2004. Credit or debit? Resource input changes population dynamics of city slicker birds. Oikos 106: 622�626. \\ \\
18. Raupp, M.J., P.M. Shrewsberry, and D.A. Herms. Ecology of Herbivorous Arthropods in Urban Landscapes. 2010. Annual Review of Entomology 55: 19-38.  \\ \\
19. Crossin, G.T., R.A. Phillips, C.R. Lattin, L.M. Romero, and T.D. Williams. 2013. Corticosterone mediated costs link current to future breeding. General and Comparative Endocrinology 193: 112-120.\\ \\
20. Harms, N.J., G.D. Fairhurst, G.R. Bortolotti, & J.E.G. Smits. 2010. Variation in immune function, body condition, and feather corticosterone in nestling Tree Swallows (Tachycineta bicolor) on reclaimed wetlands in the Athabasca oil sands, Alberta, Canada. Environmental Pollution 158: 841-848.\\ \\
21. Koren, L.,S. Nakagawa, T.Burke, K.K. Soma, K.E. Wynne-Edwards, and E. Geffen. 2012. Featherconcentrations of testosterone, corticosterone, and cortisol are associated with over-winter survival in wild House Sparrows. Proceedings of the Royal Society B 279: 1560-1566.\\ \\
22. Pearson, S., D. Levey, D., and C. Greenberg. 2003. Effects of elemental composition on the incorporation of dietary nitrogen and carbon isotopic signatures in an omnivorous songbird. Oecologia 135: 516-523. \\ \\
23 Hobson, K.A. 1999. Stable-carbon and nitrogen isotope ratios of songbird feathers grown in two terrestrial biomes: implications for evaluating trophic relationships and breeding origins. Condor 101: 799-805.\\ \\
24 Bearhop, S., R.W. Furness, G.M. Hilton, S.C. Votier, and S. Waldron. 2003. A forensic approach to understanding diet and habitat use from stable isotope analysis of (avian) claw material. Functional Ecology 17: 270-275.\\ \\
25. del Rio, C.M., P. Sabat, R. Anderson-Sprecher, and S.P. Gonzalez. 2009. Dietary and isotopic specialization: the isotopic niche of three Cinclodes ovenbirds. Oecologia 161: 149-159.\\ \\
26. Newsome, S.D, C. del Rio, S. Bearhop, and D.L. Phillips. 2007. A niche for isotopic ecology. Frontiers in Ecology and the Environment 5: 429-436.\\ \\
27. Fairhurst, G.D., A.L. Bond, K.A. Hobson, and R.A. Ronconi. 2015. Feather-based measures of stable isotopes and corticosterone reveal a relationship between trophic position and physiology in a pelagic seabird over a 153-year period. Ibis 157: 273-28.\\ \\
28. Evans, B.S., A.H. Hurlbert, and P.P. Marra. 2015b. Environmental filtering of avian communities along a rural-to-urban gradient in greater Washington, D.C. in The ecology of birds in the urban landscape: avian community composition, dispersal, and survival across the rural-to-urban gradient in metropolitan Washington DC (doctoral dissertation). pp. 3-38.\\ \\
29. Rushing, C.S., T.B. Ryder, J.F. Saracco, and P.P. Marra. 2014. Assessing migratory connectivity for a long-distance migratory bird using multiple intrinsic markers. Ecological Applications 24: 445-456.\\ \\
30. Bearhop, S., S. Waldron, S.C. Votier, and R.W. Furness. 2002. Factors that influence assimilation rates and fractionation of nitrogen and carbon stable isotopes in avian blood and feathers. Physiological and biochemical zoology 75: 451-458.\\ \\
31. Hopkins, J.B. and J.M. Ferguson. 2012. Estimating the diets of animals using stable isotopes and a comprehensive Bayesian mixing model.PloS one 7: e28478.\\ \\
32. Bortolotti, G. R., Marchant, T. A., Blas, J., & German, T. (2008). Corticosterone in feathers is a long-term, integrated measure of avian stress physiology.Functional Ecology, 22(3), 494-500.\\ \\
33. Carbajal, A., Tallo-Parra, O., Sabes-Alsina, M., Mular, I., and Lopez-Bejar, M. 2014. Feather corticosterone evaluated by ELISA in broilers: A potential tool to evaluate broiler welfare. Poultry science 93: 2884-2886.\\ \\
34. Ryder, T.B., R. Reitsma, B. Evans, and P.P. Marra. 2010. Quantifying avian nest success along an urbanization gradient using citizen and scientist generated data. Ecological Applications 20:419�426. \\ \\
35. Belaire, J.A., C.J. Whelan, and E.S. Minor. 2014. Having our yards and sharing them too: The collective effects of yards on native bird species in an urban landscape. Ecological Applications 24: 2132-2143.\\ \\
36. McCune, B. and J.B. Grace. 2002. Analysis of Ecological Communities. MjM Software, Gleneden Beach, Oregon, USA. pp. 233-256.\\ \\
37. Plummer, M. 2015. rjags: Bayesian Graphical Models using MCMC. R package version 4-4. http://CRAN.R-project.org/package=rjags\\ \\
38. Raftery, A.E. 1993. Bayesian model selection in structural equation models. Sage Focus Editions 154: 163-163.\\ \\
39. Bearhop, S., C.E. Adams, S. Waldron, R.A. Fuller, and H MacLeod. 2004. Determining trophic niche width: a novel approach using stable isotope analysis. Journal of Animal Ecology 73: 1007-1012.\\ \\

\end{document}



