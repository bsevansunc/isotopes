% Document settings:

\documentclass[12pt]{article}
\linespread{1.6}
\usepackage[margin=1.0in]{geometry}
\usepackage{graphicx}
\usepackage{setspace}
\usepackage{wasysym}

% Header information:

\singlespace \title{\Large \linespread{0} The ``urban-\emph{adapted}'' bird: Does dietary plasticity allow some species of birds to thrive in urban habitats?}
% \author{\normalsize Brian S. Evans} 
\date{}

% Start: 
\begin{document}
\maketitle \vspace{-2cm}

% Background: 

\doublespace
\noindent {\textbf {1. BACKGROUND}}: Urbanization has altered habitats, restructured avian communities, and influenced the range sizes and population dynamics of bird species (Mckinney and Lockwood 1999, Marzluff 2001, Chace and Walsh 2006). While urban environments comprise just 5.6 percent of the total land cover of the United States (US Census Bureau 2012), the shift from rural to urban land use is considered to be the second leading cause of species endangerment and extinction during the twentieth century (Czech and Krausman 1997). As the proportion of developed land is projected to increase by as much as 63 percent in the first half of this century (Lawler et al. 2014), urbanization is expected to become the primary driver of species extinction (Marzluff 2001) and thus understanding how organisms respond to urban habitats is of paramount importance. Despite the urgency of the problem, however, our ability to apply conservation strategies to minimize the effects of urbanization is hindered by a limited understanding of the mechanisms by which urban habitats influences bird populations (Shochat et al. 2006). \par

The interaction between human land use and the life history traits of species determines whether a given species will be a ``winner"�� (urban-adapted species) or ``loser"�� in human-dominated landscapes (urban-avoiding species; McKinney and Lockwood 1999; Blair 2001). It is hypothesized that bird populations and communities in anthropogenic environments are largely structured by the distribution of resources accessible to a given dietary guild (Shochat et al. 2010, Evans et al. 2015). Species with specialist dietary requirements, especially obligate insectivores (Lancaster and Rees 1979), are expected to be sensitive to human-induced habitat modification and may therefore experience high rates of local extinction across the urban habitat matrix (Devictor et al. 2007). Conversely, species with generalist dietary habitats and those that have the ability to utilize anthropogenic resources (e.g., granivores utilizing bird feeders, Robb et al. 2008) are expected to be positively associated with urban environments (reviewed in McKinney and Lockwood 1999). For example, urban landscapes dominated by ornamental and non-native species provide essential resources for frugivorous birds, especially during winter months (e.g., McCusker et al. 2010), but are expected to negatively influence insectivores due to reduced abundances of insects during the growing season (Tallamy 2004). While studies of avian community composition (e.g., Blair and Johnson 2008) and survival (e.g., Evans et al. 2015) suggest the influence of resources on avian response, the direct connection between avian health and the utilization of resources across the rural-to-urban gradient has not yet been empirically tested.\par

% RESEARCH OBJECTIVE: 

\noindent {\textbf {2. RESEARCH OBJECTIVE: To assess the influence of dietary plasticity on bird health across the rural-to-urban gradient.}}

The low temporal variability of resources in anthropogenic habitats is also hypothesized to support higher densities of individuals because they can persist in a state of reduced body condition relative to their rural counterparts (the Credit Card Hypothesis, Shochat 2004). 


Viewed cumulatively, the abundance and quality of avian food resources is expected to vary spatially along this gradient with resource subsidization as the primary driver of higher food availability at intermediate degrees of urbanization (Raupp et al. 2010).\par


Feather CORT has been found to be predictive of several biologically important indicators of bird health including breeding success (Crossin et al. 2013), body condition (Harms et al. 2010), and survival (Koren et al. 2012).

Availability of winter fruit resources on non-native and ornamental plants may provide relaxation from starvation pressures during winter months when survival of the resident species is dependent on a consistent food supply (Shochat et al. 2010).

% PROPOSED RESEARCH

\noindent {\textbf {3. PROPOSED RESEARCH}}: Here, I seek to examine the influence of urban land cover on avian diet and the effects of dietary habits on the health of birds across the rural-to-urban gradient at local and regional scales. I propose to assess variation in diet and condition across the rural-to-urban gradient in two species of birds: the Northern Cardinal (NOCA, \textit{Cardinalis cardinalis}) and Carolina Chickadee (CACH, \textit{Poecile carolinensis}). Differences and similarities in the life histories of these species, both of which are common to the rural and suburban habitats of metropolitan Washington, D.C., provide an ideal study system with which to assess the influence of urbanization on avian diet and health. Both species exhibit some degree of omnivory, are non-migratory, and utilize bird feeders in suburban and urban habitats. Key differences between the life history of the species include nesting habits (NOCA are cup nesting and CACH cavity nesting species), and response to urbanization (NOCA are found at more urban sites than CACH).
 
CORT
I would predict that the non-native plants have a negative affect on corticosterone during the summer months, due to a an expected decrease in the number of arthropods and a positive affect during the winter months due to utilization of non-native plant fruits (see Bartusveige and Gorchov 2006).

Corticosterone and isotope concentrations in feather and claw samples represent the stress and diet response of birds, respectively, at different points in a birds' annual cycle. Tail feathers are molted immediately following the end of the breeding cycle and thus CORT, $\delta^{13}$C, and $\delta^{15}$N levels represent a snapshot of the diet and health of the bird during the growth of the feather, a period of roughly two weeks (citation). Claw samples, for which only the tip of the claw will be removed, are grown continuously and samples collected at the start of the breeding season are representative of bird diet and health during the non-breeding period (winter). 


\par

Fairhurst et al. (2015) examined $\delta^{13}$C,  $\delta^{15}$N, and CORT in feathers of seabirds and found that trophic position was negatively correlated with CORT, suggesting that feeding at higher trophic positions had physiological benefits for indiviudals through increased nutrition or reduced foraging costs.


% GENERAL METHODS

\noindent \underline{GENERAL METHODS}\par

\noindent \textit {Field sampling}: This study will utilize data collected as a part of the Neighborhood Nestwatch Program (NN), a citizen science project run by the Smithsonian Migratory Bird Center. NN has established a network of over 200 sampling sites within the Greater Washington D.C. metropolitan area with sampling predominantly located at the homes of project participants. Project participants are accepted into the study based on a wide range of criteria, including level of interest, expected degree of engagement, and position along the rural-to-urban gradient. Habitats represented by study sites range from rural open and forested areas, to suburban and urban environments. From the NN network, I will select up to 50 sites with recorded CACH and/or NOCA captures, using the proportion of impervious surface to attempt to maintain as broad a representation of the rural-to-urban gradient as possible (see local and regional habitat variables, below). Technicians will visit participant properties at the start of avian breeding season (April and May). During visits, technicians will provide an educational banding demonstration and set two to eight mist nets for a period of three to five hours of mark-recapture. Birds (n = 60 of each species) will be captured using target netting with playback of mobbing calls or conspecific song. Technicians will measure body mass (to the nearest 0.01 g) on an electronic balance and unflattened wing chord (to the nearest 0.25 mm) with a wing rule. We will age individuals into juvenile and after hatch year (AHY) age classes using plumage, skull ossification, or molt criteria (methodologies vary by species, see Pyle 1997). To limit unwanted variation, only AHY male birds will be sampled. Prior to release, a  tail feather and claw sample will be taken and individuals will be marked with a US Fish and Wildlife aluminum band and a unique combination of colored plastic bands. All potential dietary sources, including samples of native  (e.g., American  Pokeweed, \textit {Phytolacca americana}) and non-native fruits (e.g., Amur Honeysuckle, \textit {Lonicera maackii}), ground and foliar arthropods, and supplemental food sources, will be collected from sites and their surrounding areas ($\leq$ 100 m) at three periods over the course of a year -- during the banding visit, at the end of the breeding season (15 July - 15 Aug), and during winter months (Dec - Feb).\par

\noindent \textit{Stable isotopes}: Samples of potential dietary sources will be dried in ovens prior to analysis. All samples will be washed in a 2:1 chloroform:methanol solution to remove surface oils and air dried in a fume hood. For each tissue and prey item, a 0.3 - 0.4 mg sample will be combusted at 1350$^{\circ}$C in an elemental analyzer and introduced into a continuous-flow isotope ratio mass spectrometer following the methods of Rushing et al. (2014). Isotopic ratios $^{13}$C and $^{15}$C f will evaluated relative to the Pee Dee Belemnite and atmospheric nitrogen standards, respectively, and reported per mil using $\delta$ notation using the equation: $$\delta{X} = \bigg[ \frac{R_{sample}}{R_{standard}} \bigg] \times 1000$$
\noindent Accuracy and precision of $\delta^{13}$C and   $\delta^{15}$N is expected to be within $\pm 2.0 \permil$ (Bearhop et al. 2002). We will use the Bayesian stable isotope mixing model  of Hopkins and Ferguson (2012) to evaluate the proportional composition of dietary sources for each individual. \par

\noindent \textit{Corticosterone}:  To examine the adult condition of birds, we will measure the stress hormone corticosterone (CORT) in feather and claw samples using a methanol based extraction, modified from Bortolotti et al. (2008), and a commercial ELISA kit (Corticosterone ELISA kit; Neogen Corporation, Ayr, UK) following Carbajal et al. (2014). The concentration of CORT in feathers and claws represent stress exposure during the time in which the each was grown and are expected to be representative of avian condition during breeding and nonbreeding periods, respectively.\par 

\noindent \textit{Local and regional habitat variables}: I will assess the variation in $\delta^{13}$C,  $\delta^{15}$N, and CORT as a function of local and regional scale habitat features. Regional habitat features include the proportion of impervious surface and canopy cover within 500 and 1000 m of a given site to approximate the position of the site along the rural-to-urban gradient. These proxy variable and scales of analysis have been shown to be predictive of a demographic response in birds to urban land cover (see Ryder et al. 2010, Evans et al. 2015). Local habitat features were sampled for NN between the years of 2010 and 2015 and include the vertical habitat complexity, canopy height and proportional cover, impervious surface, and ground cover. Additional local-scale habitat features will be assessed by participant surveys and include supplemental feeding habits (i.e., presence and maintenance of bird feeders) and the presence or absence of outdoor cats and dogs. \par

\noindent \textit{Analysis}: I will use Structural Equation Modeling (SEM; McCune and Grace 2002) to examine hypothesized relationships between habitat features, bird health, and dietary habits (below). SEM is a multivariate technique that allows us to simultaneously analyze the complex, and often correlated, inter-relationships among measured (e.g., CORT and land-cover) and latent variables (i.e., those estimated from measured values such as niche width). I will use a Bayesian framework to develop and test \textit{a priori} SEM in JAGS, implemented in Program R (see Plummer et al. 2003). I will evaluate the influence of structural linkages using Markov chain Monte Carlo sampling and use Bayesian model selection to compare \textit{a priori} models (see Raferty 1993). With SEM, I will distinguish the relative influence of local and regional scale environmental attributes on bird health and dietary traits. 

% HYPOTHESES AND PREDICTIONS

\noindent \underline{HYPOTHESES AND PREDICTIONS}\par

% HYPOTHESIS 1 plasticity

\noindent{\textbf{Hypothesis 1. The dietary habits of conspecific birds vary in composition and breadth along the rural-to-urban gradient.}} Across bioregions, urban environments have been shown to favor dietary generalists and granivores and exclude birds with insectivorous diets. Indeed, in an assessment of avian community composition across the rural-to-urban gradient in metropolitan Washington, D.C. we observed a sharp decline in the proportional abundance of obligate insectivores (Evans et al. 2015). (Note: Bearhop 2004 suggested that isotopes can be a proxy for niche width)\par

% PREDICTION 1.1 niche position

\noindent \underline{Prediction 1.1}: \textit{The ratios of $\delta^{13}$C and $\delta^{15}$N in feathers will reflect dietary shifts from insect to plant-based and anthropogenic food sources.}  Analysis of d13C and d15N in feathers (and possible diet items including bird food?) will give us a sense of the the proportional composition of plant-based foods (I’m hoping we can pick up a distinct signature for bird seed) and insects in the diet. We expect that with increasing urbanization diets will become more omnivorous and the ratio of insects to plant-based foods (including anthropogenic food sources) will decline. To test whether increased omnivory is advantageous to bird health, we use corticosterone as a proxy of avian condition. This last bit is the awkward one — however, survival was higher for NOCA in urban habitats and we assumed that this was driven by resources. Using enhanced survival of NOCA as our guide, we might expect that a species exhibiting high plasticity (NOCA) would benefit from this dietary shift while those exhibiting low plasticity (e.g., CACH) would by negatively influenced by a shift from insectivory.\par

% PREDICTION 1.2 niche width

\noindent \underline{Prediction 1.2}: \textit{ The dietary niche width among conspecifics, as measured as variance in $\delta^{13}$C and $\delta^{15}$N in feathers,  will increase with increasing urban land cover.} Populations with higher degrees of dietary specialization are expected to exhibit low variance in $\delta$-values (Bearhop et al. 2004). CChange this bit, see Newsome's paper -- Shannon-Weiner metric ... Further, we predict that the dietary niche width among NOCA will be greater than that of CACH. \par

% PREDICTION 1.3 CORT with niche width and position

\noindent \underline{Prediction 1.3}: \textit{Avian condition, as measured by corticosterone concentrations in feathers, decreases with increasing omnivory for NOCA but not CACH.} We expect ...  We will analyze CORT concentrations as a function of neighborhood and yard-scale habitat and management practices within a generalized linear model framework, using model selection to determine the variables most predictive of avian condition (see Burnham and Anderson 2002). Preliminary analysis of CORT in Carolina Chickadee (Poecile carolinensis) feathers at NN sites (n = 22) exhibit elevated CORT concentrations in sites dominated by non-native plants ($\beta$ = -3.2 $\pm$ 1.4, p-value = 0.04) but no response to the degree of urbanization at a given site (proportion of impervious surface within 500 m, $\beta$ = 0.2 $\pm$ 0.3, p-value = 0.47). These results suggest that CORT concentrations may be highly reflective of the biological response of birds to yard-scale habitat features.\par

% HYPOTHESIS 2 credit debit

\noindent{\textbf{Hypothesis 2. Urban habitats minimize starvation pressure by providing a more consistent temporal distribution of resources.}} This is a test of Shochat’s credit-debt hypothesis that we mentioned in the survivorship paper. Shochat suggested that the more natural environments have greater temporal variation which would increase starvation pressures. Under this scenario anthropogenic food sources and winter fruiting of non-native species would enhance survival. Hobson (1999) suggested that d15N enrichment is representative of nutritional stress. To test this, we would have to sample birds during the non breeding season (perhaps early spring?) and expect higher d15N ratios in rural habitats (relative to more urban ones). To test whether there is an influence of nutritional stress on bird health, we use corticosterone as a proxy of condition. \par

% PREDICTION 2.1 starvation pressure (enhanced d15N)

\noindent \underline{Prediction 2.1}: \textit{Birds will exhibit enhanced levels of $\delta^{15}$N, as measured in avian blood, along the rural-to-urban gradient.} 
text lorem ipsum yada yada. \par

% PREDICTION 2.2 CORT with starvation pressure

\noindent \underline{Prediction 2.2}: \textit{Concentrations of corticosterone in avian blood will increase proportionally with $\delta^{15}$N.} \par \par


% PROJECT RELEVANCE

\noindent {\textbf {4. RELEVANCE OF THE PROPOSED RESEARCH}}: 
My previous research has addressed the influence of urbanization on avian dispersal (dissertation citation), community composition (dissertation citation), and survival (Ryder et al. 2010, Evans et al. 2015) in metropolitan Washington, D.C. Despite emergent patterns that have clearly demonstrated the influence of urban habitat on these biological processes, the mechanisms driving these patterns remain largelly unknown. By addressing the distribution and influence of resources in avian diet, the proposed research project builds on my previous work by addressing the mechanism by which urbanization is expected to have the greatest impact on bird populations.

% ROLE OF SMITHSONIAN

\noindent {\textbf {5. ROLE OF THE SMITHSONIAN INSTITUTION}}:
Research in urban environments is often limited by accessibility and adequate coverage of the urban gradient (Cooper et al. 2007). To alleviate this problem we will utilized sites (n = 242) that were part of Neighborhood Nestwatch (hereafter referred to as “NN”), which is an ongoing citizen science project run by the Smithsonian Migratory Bird Center. NN is the ideal sampling framework for studying the effects of urbanization on avian demography because it provides access to residential properties within core urban and suburban environments as well as forested and agricultural land cover types (Figure 3.1). By incorporating privately owned land within our study design, we are able to capture portions of the urban and suburban matrix not normally monitored in avian survivorship studies.

% WORK CITED 
\noindent \begin{center} {\textbf {WORKS CITED}}\end{center} \par


\end{document}

